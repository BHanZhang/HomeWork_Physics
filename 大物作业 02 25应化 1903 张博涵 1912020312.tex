\documentclass{article}

\usepackage{ctex}
\title{大物作业}
\author{张博涵-应化1903(学号:1912020312)}
\date{\today}

\begin{document}
	\maketitle
	2、由题意得
	
	$$\vec { A } \cdot \vec { B }=(3\vec { i }+2\vec { j })\cdot(5\vec { i }+9\vec { j })=15(\vec { i }\cdot\vec { i })+(27+10)\vec { i }\cdot\vec { j }+18(\vec { j }\cdot\vec { j })$$
	
	又因为
	
	$$\vec { i } \cdot \vec { j }=0$$
	$$\vec { j }\cdot\vec { j }=\vec { i }\cdot\vec { i }=1$$
	
	所以
	
	$$\vec { A } \cdot \vec { B }=15+18=33$$
	
	3、因为$\vec { i }$、$\vec { j }$、$\vec { k }$是空间直角坐标系中三个互相垂直的矢量
	
	所以
	
	$$\vec { k }\cdot\vec { i }=0$$
	
	$$\vec { k }\times\vec { j }=-\vec { i }$$
	
\end{document}
