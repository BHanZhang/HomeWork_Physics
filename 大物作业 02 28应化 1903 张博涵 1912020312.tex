\documentclass{article}
\usepackage{ctex}
\title{大物作业}
\author{张博涵-应化1903(学号:1912020312)}
\date{\today}
\begin{document}
	\maketitle
	若记$\vec { r }=\vec { r }(x(t),y(t))$
	则有
	\begin{equation}
    x(t)=4t^2\label{1}
	\end{equation}
	\begin{equation}
    y(t)=3+2t\label{2}
	\end{equation}
	<1>将公式(\ref{2})右侧的 3 移到左边平方,再将公式(\ref{1})带入此式可得:
	\begin{equation}
    (y-3)^2=x\label{3}
	\end{equation}
	<2>对$\vec{ r }=4t^2\vec{ i }+(3+2t)\vec{ j }$的两边同时对t求导可得:
	\begin{equation}
    \vec{ v }(t)=\frac { d \vec { r } } { d t } = 8 t \vec { i } + 2 \vec { j }\label{4}
	\end{equation}
	分别代入t=1,2则得:
	\begin{center}
	$\vec{ v }(1)=8\vec { i } + 2\vec { j }$\\
	$\vec{ v }(2)=16\vec { i } + 2\vec { j }$
	\end{center}
	<3>由题目所给的条件:$\vec { r }(t)=4t^2\vec { i }+(3+2t)\vec { j }$
	分别代入t=1,2则得:
	\begin{center}
	$\vec { r }(1)=4\vec { i }+5\vec { j }$\\
	$\vec { r }(2)=16\vec { i }+7\vec { j }$
	\end{center}
	所以从 t=1 到 t=2 的位移为:
	$$\vec { x }=\vec { r }(2)-\vec { r }(1)=12\vec { i }+2\vec { j }$$
	其平均速度为:
	$$\overline{\vec { v }}=\frac{\vec{ x }}{2-1}=12\vec { i }+2\vec { j }$$
	<4>对(\ref{4})两边再同时对t求导得:
	$$\vec{ a }=\frac{d\vec{r}}{dt}=8\vec{ i }$$
	则可算出加速度的总大小为$ \| \vec { a } \|=8$\\
	速度的大小也可算出:$v=\| \vec { v } \|=\sqrt { ( 8 t ) ^ { 2 } + 4 }=2\sqrt{16t^2+1}$
	则切向加速度可算得:$$a_t=\frac{dv}{dt}=\frac{2d\sqrt{16t^2+1}}{dt}=\frac{32t}{\sqrt{16t^2+1}}$$
	则由加速度大小的表达式$a=\sqrt{a _ { n } ^ { 2 }+a _ { t } ^ { 2 }}$\\
	得$$a_n=\sqrt{a^2-a_{t} ^{2}}=\sqrt{32}\sqrt{2-\frac{32t^2}{16t^2+1}}=\frac{8}{\sqrt{16t^2+1}}$$
\end{document}




	