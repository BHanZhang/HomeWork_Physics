\documentclass{article}
\usepackage{ctex}



\title{大物作业}
\author{张博涵-应化1903(学号:1912020312)}
\date{\today}

\begin{document}
	\maketitle
	
	<1>\quad D \\
	
	<2>
	(1)将弹簧由$x_1=0.5m$拉伸至$x_2=1m$过程中,有外力所作功与弹簧势能变化量加和为 0,则外力所做之功:
	$$W_{12}=-\int _ {0.5} ^ {1.0}\vec{F} \cdot d\vec{x}(J)$$
	
	
	
	$$ \quad \quad \quad \quad \quad \quad \quad =\int _ {0.5} ^ {1.0}(42x+36x^2)x\cdot \vec{i}\cdot \vec{i} (J)$$
	
	
	
	$$ \quad \quad \quad \quad\quad \quad \quad \quad \quad \quad \quad =\int _ {0.5} ^ {1.0}(42x+36x^2)x dx(J)=26.25(J)$$\\
	(2)力的方向与位移的方向共线,那么取这个方向所在直线上的任一条曲线$\Gamma$,必然可以切分为$\Gamma_1$和$\Gamma_2$其中$\Gamma_1$对应 $x$从 $0$ 到$x_0$,$\Gamma_2$对应 $x$从 $x_0$ 到$x$
	则$$\oint_{\Gamma}\vec{F} \cdot d\vec{x}(J)= 
	-\int _ {0} ^ {x_0}\vec{F} \cdot d\vec{x}(J)-\int _ {x_0} ^ {0}\vec{F} \cdot d\vec{x}(J)=-\int _ {0} ^ {x_0}\vec{F} \cdot d\vec{x}(J)+\int _ {0} ^ {x_0}\vec{F} \cdot d\vec{x}(J)=0$$
	则弹力是保守力.

	
	
	
\end{document}